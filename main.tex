\documentclass[a4paper,12pt]{article}

\usepackage[utf8]{inputenc}
\usepackage[T1]{fontenc}
\usepackage[ngerman]{babel}
\usepackage{hyperref}
\usepackage[a4paper, left=2cm, right=5cm, top=2cm]{geometry}
\renewcommand{\baselinestretch}{1.5}
\usepackage{cite}
\usepackage{bibgerm}
\setcounter{secnumdepth}{4}

\author{Tobias Sigmann}
\title{Seminararbeit: Lorawan}
\date{\today}

\begin{document}
    \maketitle
    \newpage
    \tableofcontents
    \newpage
    \begin{abstract} 
       \cite{WhatIsLoRa}
       \newline
       \cite{Semtech}
    \end{abstract}

    \section{Einführung in Lora}
        Lora ist ein Low Power, Wide Area (LPWA) Netzwerkprotokoll und somit sehr gut für batteriebetriebene kabellose Geräte geeignet. Deswegen wir Lora auch of im Internet of Things (IoT) Bereich verwendet.
        Mittels der bidirektionalen Kommunikation ist es möglich Daten und Befehle über weite Strecken zu übertragen. Leider leidet darunter die Geschwindigkeit, sodas sich Lora nicht als WLAN Ersatz eignet.
        Trotzdem können zwischen 0.3 und 50 kbps erreicht werden. In Europa werden 863 MHz bis 870 MHz verwendet. Allerdings variiert der Frequenzbereich für andere Kontinente. Je nach Bedingungen können so bis zu 20km entfernte Endgeräte erkannt und mit diesen kommuniziert werden.
        Es ist sogar möglich den Standort des Gerätes zu bestimmen.
        
        Eine Alternative zu Lora ist Sigfox, hierrauf werde ich nicht weiter eingehen.
        LoRaWAN 1.1
        \newline    
        \cite{WhatIsLoRa}(Optimiert für Batterie Kapazität(Teilnehmer) Reichweite, Kosten
        mehrjährige Batterielaufzeit, kleine Datenmengen, große Reichweite, LPWAN (Low Power WAN)
        
        Kriterien für Lora: Netzwerk Architektur, Reichweite, Batterielaufzeit, Interreferenzrobustheit, Anzahl Konten, Sicherheit, bidirektionale Kommunikation, verschiedene Anwendungsunterstützung
        
        Orientiert für Mobile Adressierbare Endgeräte)
        \newline{}
            \cite{LoraLimit}(
                alternativen: Sigfox, Ingneu, Dash7

                Klassen Kompromiss zwischen Reichweite, Performance(Latzen/ Durchsatz) und Energiebedarf

                Energiesparend durch ADR (Adaptive Daten Rate)
            )
        \newline{}\newline{} Es wird folgen: Was ist lora, wo und wofür wird es benutzt, wie weit kann man senden und wie schnell...
    \section{Aufbau eines Lora-Netzwerk}
        \cite{WhatIsLoRa}(Architektur hat größen Einfluss auf Batterie, anz Teilnehmer, qualität, sicherheit,
        
        )
        \cite{LoRaSpec}(
            Geschwindigkeit ist kompromiss zwischen abstand/geschw. die untersch freuenzen bze. geschwindigkeiten beinflussen sich nicht gegeseitig => keine inteferenz
        )
        \subsection{Gateway}
            Um die mittels Lora übertragenen Daten weiterzuverarbeiten, ist ein Gateway nötig, das über Lora empfangenen Daten an einen im Internet befindlichen Server sendet. Dies wird möglich, indem das Gateway die RF Pakete in IP/TCP Pakete umwandelt.
            Die Endgeräte kommunizieren direkt mit dem Gateway (Single-Hop-Connection) und stellen somit eine Sterntopologie her.

            Ein Endgeräte kann gleichzeitig an mehren Gateways senden. Dabei sind die Endgeräte in Multicast Gruppen unterteilt.
            \cite{TheThing}
            \cite{WhatIsLoRa}(
                Meistens wird ein Netzförmiges Netzt aufgebaut. Knoten leiten Nachrichten weiter => größere Reichweite aber kompliziert, erlaubt weniger Teilnehmer und energieaufwändig). 

                Lora Sternförmig => Energie-effizient, Knoten senden direkt an Gateways. Gateways senden an Server, Server muss doppelte Pakete filtern, Sicherheitscheck, ACK über bestes Gateway senden, datenrate anpassen.

                Keine Handover

                Gateway müssen viele Geräte handeln da Stern. erreichen durch (adaptive Datenrate, multi chanel/multi modem transive) mehrere Nachrichten auch verschieden Chanels gleichzeitig empfangen

                Wichtige Faktoren(anz. chanels, datenrate(time on air), paylod länge, Sendehäufigkeit)
            
                Skaliert sehr gut => gemacht für große Nutzerzahlen
                Neues gateway kann Knoten 6-8 x verbessern
                )
            \cite{LoRaSpec}(
                Applikation Server -> Zentraler Server(leitend Pakete weiter) -> Gateway(wandelt lorawan in ip Pakete um) -> Endgerät/Knoten       )
            \cite{RFC8376}
            \newline{}\newline{} Es wird folgen: Wie baut man ein Netzwek mit einem Gateway auf. Wie greift man auf die übermittelten Daten zu ...
        \subsection{Netzwerkbeitritt}
        \cite{LoRaSpec}(wei arten OTAA(Over the air activation), ABP(Activation by Personalization)
                Jedes gerät hat eine vorgegebene DevEUI (wie MAc adresse eines PCs), JoinEUI muss angegeben werden und adressiert den Join server \marginpar{Was macht der? wohin damit?}
                
        )
    \section{Lora Funktionsweise}
        Die Datenrate ist einstellbar, jedoch wird die Reichweite bei höherer Datenrate gemindert. Ein Vorteil von Lora ist, das die einzelnen Datenraten nicht interferieren und so jedes Endgerät seine eigene Datenrate unabhängig von den anderen Verwenden kann.
        Außerdem wird die Datenrate und die Sendeleistung für jedes Gerät separat gesteuert (ADR, Adaprive Data Rate)
            \cite{RFC8376}
            \cite{LoRaSpec}
            \cite{WhatIsLoRa}()
        \subsection{Schichtenmodell}

            \cite{WhatIsLoRa}(

            Application

            Lora MAC

            MacOptins(Classes)

            LoraModialtion(Regionales ISB)
            )
            Lora: Long Range ist Physikalische schicht(gibt an wie die kabellose übertragung geschieht)
            LoraWan:protokoll das.
        \subsection{Protokoll}
            \cite{LoRaSpec}(
                Mac commands werden benutzt um geräte zu steuern => frequenzen zu ändern, ...
                Application wird diese nie erhalten, läuft zwischen netzwerkserver und lora gerät ab.
                Verschlüsselt hier oder da.
                aufbau: 1byte command, x byte extra data.
                müssen vom empfancher acknolaged werden. Reihenfolge ist zu be<chten. Alle nahrichten in einem farme müssen auch in einem frame ack werden. => Macbuffer ermöglicht dies. Wenn buffer überleuft werden die ältesten ack. \marginpar{(Was passiert mit dem rest?}
            )
        \subsection{Übertragungsart}
            \cite{LoRaSpec}(
                Knoten können zu jeder Zeit, auf beliebigen Kanälen, beliebig schnell, beliebig lange senden, solange folgende regeln befolgt werden.
                \begin{itemize}
                    
                        \item Channels werden per Pseudozufallszahl geändert
                        \item Sendezeit erfüllt die Regionalen Bestimmungen
                    
                \end{itemize}
            )

            \cite{LoraLimit}(
                Chrip Signal => Zeitliche Änderung in Trägerfrequenz(höhere Frequenz als Datenrate)(positiv chrip/negativ chrip)

                Datensignal wird in Chrip Signal moduliert. Resultierende Signal ist breitbandiger als Datensignal. Maximale Datenrate auch mit Rauschen erreichbar.

                Durch orthogonale "Spread Factor" mehrere Signale auf einem Chanel
            )
            \cite{WhatIsLoRa}(normal FSK, schon sehr efficent. Lora "chirp spread spectrum odulation". Ist wie FSk aber größere Rechiweite, robuster. Stammt aus dem Militär/raumfahrt.Lora als erstes für kommerziellen billigen Einsatz.
            
            Spread spectrum => signale sind Ortohonal für versch. spreizraten, fakto koreliert mit datenrate => verschiedene Datenraten auf einem Kanal
            
            Nähere Geräte sind schneller => höhere Datenrate => kürzee übertrgungsdauer und lassen somit merh zeit für andere, => bessere Batterielaufzeit. Deswegen sidn symetrische up/downlinks nötig.)
            Frequenzhopping, spread spectrum, code-chanels
                \subsubsection{Adaptive Data Rate}
                \cite{LoraLimit}(
                    Datenrate und Funkfrequenz(RF) werde passend zum Abstand angepasst

                    nahe Knoten => hohe Datenrate => kurze Sendezeit => weniger RF-Power
                    kann nach Bedarf geändert werden

                    => immer möglichst schnelle senden => weniger Energie

                )
                \section{Notitz: protokoll/paket hierfor erklären}
    \section{Lora Geräte Klassen}
        \cite{LoRaSpec}(GGeräte müssen mindestens A können, alle die mehr können werden auch "high class End-Devices" genannt)
        Vieleicht zu klein => in anderes Kapitel stopfen. Bei mehrfacher übertrageung wird nicht erhöht

        Die Endgeräte sind je nach Kommunikationsart/Protokoll Art in drei Klassen (A, B und C) unterteilt. 
        
        Jede Klasse hat 3 counter FCntUP(Pro uplink ++), FCNTDown(pro downlink auser port 0 => mach), AFCntDOwn(port ungleich 0 dann ++)   (nur beschreiben wie diese grob funktionieren) Zähler sollen nich flüchtig sein(Batteriewechseln kein reset)
        bei neuverbinden müssen alle counter auf uf 0 gesetzt werden. counter müssen auf beiden seiten glich gehalten werden(Synchron geführt)
        Wenn nachricht empfagen ist muss der darin enthaltenene counter größer sein als der eigene. 

        die Counter Werte sollen so weit wie möglich nur einam verwendet werden.
        
        )
        \cite{RFC8376}
        \cite{LoraClasses}
        \cite{WhatIsLoRa}(Asynchrone Knoten wegen Batterie => Event/Scheduler gesteuert verwendet ALOHA
        
        Normal Netze müssen sich synchronisieren und Nachrichten abrufen. Lora partiell nicht => laut GSMA 3 bis 5 fach effizienter)
    
        zur besseren Anpassung/ Anpassung an Batterie
        
        EU: 10 Kanäle (8: 250bps bis 5.5kbps) (1: FSK 50kbps) (high rate Lora 114kbps)

        )
        \subsection{Klasse A}
            Klasse A zeichnet sich durch sehr geringer Stromverbrauch aus. Die Kommunikation kann bidirektionalen Stadtfinden, allerdings muss die Kommunikation von dem Endgeräte gestartet werden. Das bieted die möglichkeit das das Endgerät, wenn keine Daten gesendet werden müssen, in einen sehr
            sparsamen Schlafmodus wechselt. Um das Endgeräte nicht zum "aufwachen" zwingen zu müssen, wurde auf einen "Hardbeat" oder ähnliches verzichtet. Das Endgerät kann so lange "schlafen" wie es möchte. Dadurch ist die Klasse A auch die potenziell Stromsparende Endgeräteklasse.
            
            Das Endgerät startet die Kommunikation in dem es Daten an das Gateway sendet(uplink). Daraufhin hat das Gateway die Möglichkeit 2 mal Daten zum Endgeräte senden(downlink). Da die Kommunikation asynchron Stadtfinden wartet das Endgeräte bis es beide uplinks empfangen hat.gewünscht.
            
            Um zu ermitteln, wann gesendet werden darf, wird das ALOHA-Protokoll verwendet.
            Da das Gateway nicht immer Daten an die Endgeräte senden kann, muss es diese zwischenspeichern um diese bei der Nächsten Kommunikation zum senden.
            \cite{LoRaSpec}(Es wird zwischen up- / downlink unterschieden)

            \subsubsection{Uplink}

                \cite{LoRaSpec} 
                    (vom Knoten zum Gateway(1..n), Lora radio packet mode, (Preamble, PHDR, PHDR\_CRC, PHYPayload, CRC(cyklische Rendundanz Prüfung))
                 )

            \subsubsection{Downlink}

                \cite{LoRaSpec}
                (
                    radio packet explicit mode, vom Gateway(1) zum Knoten(1), ausgelöst vom Netzwerkserver, auch multikasts möglich, (Preamble, PHDR, PHDR\_CRC, PHYPayload)
                Um Nachricht kurz zu halten kein CRC am ende, nach Reciever\_Delay1 / Reciever\_Delay2 kann empfangen werden (rx1, rx2) 
                
                Fenster müssen lange genug für Preamble auf bleiben=> wenn erkannt wird empfangen wenn nicht fenster weider zu. Es darf nur gesendet werden wenn beide fenster zu sind.
                Es ist auch erlabut andere protokole zu sprechen wenn ncht gesendet oder gehört wird.
                )
                \paragraph{Rx1}
                    \cite{LoRaSpec}(
                        Frequenz abhänig von Uplinkfrequenz, Datenrate abhängig on Uplinkdatenrate, wird nach Reciever\_Delay 1 +/- 20 msec erwarted, Datenrate auch abhänig von Regionalen regeln, Standart: Datenrate = Uplinkdatenrate
                    )
                \paragraph{Rx2}
                    \cite{LoRaSpec}(
                        feste Frequenz/Dtanetnrate, nach Delay2 +/- 20 msec, Frequenz/Datenrate mittels MAC änderbar
                    )
                    \cite{LoRaSpec}(
                        Öffnungslänge muss für Preamble ausreichen, nach RX1 + MIC und autentigitätscheck muss nicht zwingen RX2 geöffnet werden, Sender muss in einem der beiden Fenster stattfinden, Falls Downlink über beide Fenster => feames müsen gleich sein. Knoten dürfen nich während empfangen/ zwischen RX1 und RX2 senden, ender Protokolle dürfen gesprochen werden wenn gesendet werden darf

                        )
        \subsection{Klasse B}
            Die Klasse B bietet bidirektionale Kommunikation mit einer deterministischem downlink Latenz. Um diese latenz zu gewährleisten, muss die Kommunikation Synchron ablaufen. Außerdem muss festgestellt werden, ob das Endgerät bzw das Gateway noch in Reichweite ist. Dies wird mittels einens periodischem "beacon" die zu festgelegten
            Zeitpunkten gesendet werde realisiert.

            Die Latenz ist einstellbar und kann bis zu 128 Sekunden.

            Obwohl das Endgerät durch die periodischen "beacons" nicht "schalfen" kann, ist die Klasse B für den Batteriebetrieb gedacht.
            \cite{LoRaSpec}(wird verwendet wenn mehr bedarf für empfangsfenster ist. Hierzu ist ein synchronsignal nötig=> zu bestimmten zeiten kann damit empfangen werden
                Gateway sended Beacon für synchrinsation. Um daten empfangen zu werden werden empfangsslots => pingslots verwendet, werden periodisch geöffneto und mittels beacon synchronisiert. Normalerweise werde diese schnell geschlossen außer es wird etwas empfagne.
                Gateway dessen beacon benutzt wird, wird nach empfangsqualität ausgewählt. Wenn neuer/unbekanter Beachon von einem anderen GAteway empfangen wird, wird der netzwerkserver benachrichted und dieser entscheidet welcher verwendet wird()passt rozte an).

                Das Netzwerk muss die standart ping-slot periode Datenrate und kanal kennen.

                Um ein gerät auf klasse B zu kommen muss erst von Klasse A gewächselt werden.

                Entgeräte müssen Netzwerkserver über position nformieren. Dies kann über eine leere nachricht passieren oder eine normale(uplink).
                
                Das beacon und die enthaltenen daten werden an die applikation geschiht. Der server kann den beacon auswerten. ziwschen beacon und uplink wird random time verwendet um kolisionen zu verhindern .\marginpar{Nachschuen wie genau das funktioniert}
                änderungen an pingslotperiode .. muss mitgeteilt werden. Hierzu ist klasse A nötig => wechel zu A, wechel zu B.
            )
            \subsubsection{Klassenwechsel A nach B}
                \cite{LoRaSpec}(
                    Endgerät fart LoRaWAN layer an. LAyer sucht beacon. MAc command DeviceTImeReq um schneller bacon zu bekommen nutzen. Danach wird das ClassB feld auf 1 gesetzt. Bei den geöfneten fenstern wrd der maximal mögliche clockdrift berücksichtigt.
                    Downlink läuft wie bei A ab.


                )
            \subsubsection{Uplink}
            \cite{LoRaSpec}(Wie bei)
            \subsubsection{Downlink}
            \cite{LoRaSpec}(wie bei A, frequenzplan kann sich unterschieden. AUch Multikasts möglich,)        
        \subsection{Klasse C}
            Um eine möglichst geringe/keine Latzen zu erzielen ist die Klasse C gemacht. Dies bedeutet aber auch das der Stromverbrauch am höchsten ist und somit nicht für den Batteriebetrieb geeignet.
            Das Gateway kann immer Daten senden außer wenn das Endgerät gerade Daten sendet. Hier sind Geschwindigkeit von bis zu 50mb möglich.

            Es ist auch möglich während des Betriebes eines Endgerätes die Klasse zu wechseln. Dies wird am häufigste zwischen A und B getan/ ist nur zwischen A und B möglich.
    \section{Sicherheit}
        Lora bietet die end-to-end Sicherheit an, indem es die Signale zweimal verschlüsselt.

        Die erste Verschlüsselung dient dazu die gesendeten Daten vor eventuellen Mithörern zu verschlüsseln. Die Verschlüsselung geschieht mit einem 128-bit Network-Session-Key.

        Die zweite Verschlüsselung wird bis zur endgültigen Weiterverarbeitung der Daten auf z.B. einen Server verwendet und ist ein 128 bit Application-Session-Key.

        Das zur Verschlüsselung verwendete Protokoll ist AES. Auch zu Authentifizierung und zur Überprüfung der Integrität wir AES verwendet.
        \cite{LoRaSecur}
            \cite{RFC8376}
            \cite{WhatIsLoRa}(Applikationsverschlüsselung(schutz der Daten for mitlesen) Netzwerk(Autentiizierung der Knoten) AFS, Key Exnage IEEE EU164)
            \cite{LoRaSpec}(symetrischer Schlüssel => nur einer benötigt, Sessionkey ist abgeleited von Knoten-rootkey. JoinServer setllt verbindung der Keys her.
            )
    \section{Live-Beispiel}
        wenn vorhanden.

    \section{Fazit}
    \section{Sonstige quellen}
        \url{https://lora-alliance.org/resource-hub}
        QuickStart für ein kleines Projekt: \url{https://www.thethingsnetwork.org/docs/devices/node/quick-start.html#setup-arduino-ide}

        Weitere Infos \url{http://www.multitech.com/documents/publications/marketing-guides/lora_device_dev_guide_orange.pdf}
        \newpage
    \bibliographystyle{geralpha}
    \bibliography{myBib}
    %richtige namen finden
\end{document}