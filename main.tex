\documentclass[a4paper,12pt]{article}

\usepackage[utf8]{inputenc}
\usepackage[T1]{fontenc}
\usepackage[ngerman]{babel}
\usepackage{hyperref}
\usepackage[a4paper, left=2cm, right=5cm, top=2cm]{geometry}
\renewcommand{\baselinestretch}{1.5}
\usepackage{cite}
\usepackage{bibgerm}

\begin{document}
    \tableofcontents
    \newpage
    \begin{abstract} 
       \cite{WhatIsLoRa}
       \cite{Semtech}
    \end{abstract}

    \section{Einführung in Lora}
        \cite{WhatIsLoRa}
        \cite{LoraLimit}
    was ist lora, wo und wofür wird es benuzt, wie weit kann man senden und wie schnell s...
    \section{Aufbau eines Lora-Netzwek}
    \subsection{Gateway}
        \cite{Loriot}
        \cite{TheThing}
        \cite{WhatIsLoRa}
        \cite{LoRaSpec}
        \cite{RFC8376}
    Wie baut man ein Netzwek mit einem Gateway auf. Wie greift man auf die übermittelten Daten zu ...
    \subsection{??End-Gerät zu End-Gerät}
    Sehr wenig quellen. Uhrprünglich nicht vorgesehen
    \section{Lora Funktionsweise}
        \cite{RFC8376}
        \cite{LoRaSpec}
    \subsection{Schichtenmodell}
    \subsection{Protokoll}
    \subsection{Übertragungsart}
    hardware
    \section{Lora Geräte Klassen}
        \cite{RFC8376}
        \cite{LoraClasses}
    \subsection{Class A}
    \subsection{Class B}
    \subsection{Class C}
    \section{Sicherheit}
        \cite{LoRaSecur}
        \cite{RFC8376}
        \cite{WhatIsLoRa}
    \section{Sonsitge quellen}
    \url{https://lora-alliance.org/resource-hub}

    \newpage
    \bibliographystyle{geralpha}
    \bibliography{myBib}
    %richtige namen finden
\end{document}